\documentclass[11pt,letterpaper,reqno]{amsart}
\usepackage{tikz}
\usetikzlibrary{positioning, shapes.geometric, arrows.meta, calc, positioning}
\usepackage{amssymb}
\usepackage{amsmath}
\usepackage{amsthm}
\usepackage{amsfonts}
\usepackage{bbm}
\usepackage{enumitem} 
\usepackage{pgfplots}
\pgfplotsset{compat=1.18} 
\usepackage{booktabs}
\usepackage{graphicx}
\usepackage[T1]{fontenc}
\usepackage{doi}
\usepackage{float} 
\usepackage{comment} 
\addtolength{\hoffset}{-1.5cm}\addtolength{\textwidth}{3cm}
\addtolength{\voffset}{-1cm}\addtolength{\textheight}{2cm}

\usepackage{bookmark}
\usepackage{hyperref}
\hypersetup{pdfstartview={FitH}}
\newcommand{\C}{\mathbb{C}}
\newcommand{\cE}{\mathcal{E}}
\newcommand{\norm}[1]{\lVert #1 \rVert}
\newcommand{\abs}[1]{| #1 |}
\newcommand{\bv}{\mathbf{v}}
\newcommand{\bw}{\mathbf{w}}
\newcommand{\tr}{\operatorname{Tr}}
\DeclareMathOperator{\rank}{rank}




\newtheorem{theorem}{Theorem}[section]
\newtheorem{lemma}[theorem]{Lemma}
\newtheorem{proposition}[theorem]{Proposition}
\newtheorem{corollary}[theorem]{Corollary}
\newtheorem{claim}{Claim}
\newtheorem{question}[theorem]{Question}
\newtheorem{problem}[theorem]{Problem}
\newtheorem{conjecture}[theorem]{Conjecture}
\theoremstyle{definition}
\newtheorem{example}[theorem]{Example}
\newtheorem{remark}[theorem]{Remark}
\newtheorem{definition}[theorem]{Definition}
\numberwithin{equation}{section}
\newcommand{\NN}{\mathbb{N}}
\newcommand{\taufunc}{\tau}
\newcommand{\omegap}{\omega}
\newcommand{\ord}{\operatorname{ord}}
\newcommand{\R}{\mathbb{R}}        % real numbers
\newcommand{\E}{\mathbb{E}}        % expectation
\newcommand{\Var}{\mathrm{Var}}    % variance
\newcommand{\Cov}{\operatorname{Cov}}
\newcommand{\PP}{\mathbb{P}}     % probability
\newcommand{\eps}{\varepsilon}     % epsilon
\newcommand{\ind}{\mathbf{1}}      % indicator function
\newcommand{\seq}[1]{\left(#1\right)} % sequence
\newcommand{\lcm}{\operatorname{lcm}}
\makeatother


\begin{document}

% \maketitle



\section{Introduction}
In~\cite[p.~40]{ErGr80}, Erd\H{o}s and Graham posed the following question, which also appears as Problem~\#311 on Bloom's Erd\H{o}s Problems website~\cite{EP}.
\begin{problem}
Let $\delta(N)$ be the minimal value of $\lvert 1-\sum_{n\in A}\frac{1}{n}\rvert$ as $A$ ranges over all subsets of $\{1,\ldots,N\}$ which contain no $S$ such that $\sum_{n\in S}\frac{1}{n}=1$? Is it\[e^{-(c+o(1))N}\]for some constant $c\in (0,1)$?
\end{problem}

It is trivially at least \(1/\mathrm{lcm}(1,\ldots,N)\).

In this note, we use a representation lemma of Liu--Sawhney~\cite{LiSa24} to obtain the following nontrivial upper bound: there exists \(c_0>0\) such that for all sufficiently large \(N\),
\[
\delta(N)\le \exp \left(-c_0 \frac{N}{(\log N)^3(\log\log N)^4}\right).
\]



\section{A nontrivial upper bound for $\delta(N)$ via Liu--Sawhney}

\subsection{Definition}

Throughout, $\log$ denotes the natural logarithm.
For a positive integer $N$, define
\[
\delta(N):=\min\left\{\left|1-\sum_{n\in A}\frac1n\right|:\ 
A\subseteq\{1,\dots,N\}\ \text{and $A$ is admissible}\right\},
\]
where we say that $A$ is \emph{admissible} if it contains no subset $S\subseteq A$ with
$\sum_{n\in S}\frac1n=1$.
Note that any $A$ with $\sum_{n\in A}\frac1n<1$ is automatically admissible, since then
$\sum_{n\in S}\frac1n<1$ for every $S\subseteq A$.

We say that a positive integer $n$ is \emph{$S$-smooth} if every prime
power $q|n$ satisfies $q \leq S$.

\subsection{Two standard inputs}

\begin{lemma}[Liu--Sawhney {\cite[Lemma~4.1]{LiSa24}}]\label{lem:LS}
There exist absolute constants $c_{\mathrm{LS}}>0$ and $N_{\mathrm{LS}}\in\mathbb N$
such that the following holds for all $N\ge N_{\mathrm{LS}}$.
Let
\[
S:=c_{\mathrm{LS}}\frac{N}{(\log N)^3(\log\log N)^4}.
\]
If $t$ is \emph{$S$--smooth}, and $s$ is an integer with
\[
\frac{t}{3}\le s\le t,
\]
then there exists a subset $A\subset [N/16,N]\cap\mathbb N$ such that
\[
\sum_{n\in A}\frac1n=\frac{s}{t}.
\]
\end{lemma}

\begin{lemma}\label{lem:lcm-psi}
Let
\[
\psi(x):=\sum_{p^k\le x}\log p
\]
be Chebyshev's function. Then for every real $x\ge 1$,
\begin{equation}\label{eq:lcm-psi}
\log\mathrm{lcm}(1,2,\dots,\lfloor x\rfloor)=\psi(x).
\end{equation}
Moreover, by the prime number theorem (equivalently $\psi(x)\sim x$), there exist absolute
constants $x_0\ge 2$ and $c_\psi>0$ such that for all $x\ge x_0$,
\begin{equation}\label{eq:psi-lower}
\psi(x)\ge c_\psi x.
\end{equation}
\end{lemma}

\subsection{Main theorem}

\begin{theorem}\label{thm:nontrivial-upper}
There exist absolute constants $c_0>0$ and $N_0\in\mathbb N$ such that for all $N\ge N_0$,
\[
\delta(N)\ \le\ \exp\!\left(-c_0\,\frac{N}{(\log N)^3(\log\log N)^4}\right).
\]
\end{theorem}

\begin{proof}
Let $N\ge N_0$ be sufficiently large, where $N_0$ will be chosen at the end.
Define
\[
S:=c_{\mathrm{LS}}\frac{N}{(\log N)^3(\log\log N)^4},\qquad
S_0:=\lfloor S\rfloor,\qquad
t:=\mathrm{lcm}(1,2,\dots,S_0),\qquad
s:=t-1.
\]
Since every prime power divisor of $t$ is at most $S_0\le S$, the integer $t$ is $S$--smooth.
Also, once $S_0\ge 2$ we have $t\ge 2$, hence $t/3\le t-1=s\le t$.

Applying Lemma~\ref{lem:LS} to these parameters $(s,t)$, we obtain a set
\[
A\subset [N/16,N]\cap\mathbb N\subseteq \{1,2,\dots,N\}
\]
such that
\[
\sum_{n\in A}\frac1n=\frac{s}{t}=1-\frac1t.
\]
In particular $\sum_{n\in A}\frac1n<1$, so $A$ is admissible for the definition of $\delta(N)$.
Therefore
\begin{equation}\label{eq:fN-by-t}
\delta(N)\ \le\ \left|1-\sum_{n\in A}\frac1n\right|=\frac1t.
\end{equation}

It remains to lower bound $t$.
By \eqref{eq:lcm-psi} and \eqref{eq:psi-lower}, provided $S_0\ge x_0$ we have
\[
\log t=\log\mathrm{lcm}(1,2,\dots,S_0)=\psi(S_0)\ge c_\psi S_0.
\]
If additionally $S_0\ge 2$, then $S_0\ge S/2$, hence
\[
\log t\ \ge\ c_\psi S_0\ \ge\ \frac{c_\psi}{2}\,S.
\]
Consequently,
\[
\frac1t\ \le\ \exp\!\left(-\frac{c_\psi}{2}S\right)
=\exp\!\left(-\frac{c_\psi c_{\mathrm{LS}}}{2}\,\frac{N}{(\log N)^3(\log\log N)^4}\right).
\]
Combining this with \eqref{eq:fN-by-t} gives the claimed bound with
$c_0:=\frac{c_\psi c_{\mathrm{LS}}}{2}$.

Finally, choose $N_0$ large enough so that simultaneously:
(i) $N\ge N_{\mathrm{LS}}$ (to apply Lemma~\ref{lem:LS}),
(ii) $S_0=\lfloor S\rfloor\ge x_0$ (to apply \eqref{eq:psi-lower}),
and (iii) $S_0\ge 2$ (so that $t\ge 2$ and $S_0\ge S/2$).
This completes the proof.
\end{proof}













\begin{thebibliography}{9}

\bibitem{EP}
T. F. Bloom, Erd\H{o}s Problem \#311, \url{https://www.erdosproblems.com/311}, accessed 2026-01-12.

\bibitem{LiSa24}
Y. P. Liu, M. Sawhney, On further questions regarding unit fractions. arXiv preprint arXiv:2404.07113. \url{https://arxiv.org/abs/2404.07113v1}


\bibitem{ErGr80}
Erd\H{o}s, P. and Graham, R., Old and new problems and results in combinatorial number theory. Monographies de L'Enseignement Mathematique (1980).

\end{thebibliography}

\end{document}
